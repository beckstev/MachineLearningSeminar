\setcounter{page}{1}
\section*{Motivation}
Nach eigener Umfrage schätzen Physikstudierende (verschiedenster Semester)
an der TU Dortmund, dass sie etwa $17 \pm 12$ Hunderassen voneinander
unterscheiden und benennen können.
Auf Grund der geringen Teilnehmerzahl von $35$ Befragten
ist die Aussagekraft mit Sicherheit begrenzt. Eine Histogrammierung
der Antworten findet sich im Appendix \ref{fig:Antwortverteilung}.
Jedoch kann sie verwendet werden, um die Hypothese zu motivieren:
\begin{center}
  Nur ein gegen Null gehender Anteil der Physikstudierende schafft es $120$
  Hunderassen bennen zu können.
\end{center}
Wird davon ausgegangen, dass die Teilnehmer immer wenn ihnen ein Bild einer bekannten
Hunderassen gezeigt wird dieses richtig klassifizieren,
entspricht der obige Mittelwert einer mittleren Genauigkeit von
\begin{equation}
  \label{eq:umfrage_mittlere_genauigkeit}
  \ov{\map{acc}}\ua{umfrage} = 14 \pm 10 \%.
\end{equation}
Das Ziel dieses Projekt wird es sein \emph{genauer} als ein durchschnittlicher
Physikstudierender zu sein.
Hierfür werden verschiedene Neuronale Netzwerke entwickelt, optimiert und getestet.
Zusätzlich wird als alternativer Ansatz ein Random Forest als Klassifizier verwendet.
