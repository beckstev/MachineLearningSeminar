\setcounter{page}{1}
\section*{Motivation}
Nach eigener Umfrage schätzen Physikstudierende (verschidenster Semester)
an der TU Dortmund, dass sie etwa \textbf{NummerRassen} Hunderassen voneinander
unterscheiden und benennen können
Auf Grund der geringen Teilnehmerzahl von \textbf{Anzahl der Gefragten}
ist die Allgemeinheit der Umfrage mit Sicherheit begrentzt. Jedoch kann sie
verwendet werden um die Hypothese zu motivieren:
\begin{center}
  Nur ein gegen Null gehender Anteil der Physikstudierende schafft es $120$
  Hunderassen bennen zu können.
\end{center}
Gehen wir davon aus, dass die Teilnehmer wenn immer ihnen ein Bild einer bekannten
Hunderassen gezeigt wird richtig bennen, entspricht der obige Mittelwert eine
mittlere Genauigkeit von
\begin{equation}
  \label{eq:umfrage_mittlere_genauigkeit}
  \ov{\map{acc}}\ua{umfrage} = ????.
\end{equation}

Das Ziel dieses Projekt wird es sein \emph{genauer} als ein durchschnittlicher
Physikstudierender zusein.
Hierfür werden verschiedene Neuronale Netzwerke entwickelt, optimiert und getestet.
Zusätzlich wird als Alternativeransatz ein Random Forest als Klassifizier verwendet.
