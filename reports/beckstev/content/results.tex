\section{Ergebnisse}
Im folgenden Kapitel soll Testresultate der einzelnen Netzwerke präsentiert
werden. Als Leistungsmaß wird die Genauigkeit des jeweiligen Modells präsentiert.
Hierfür wird neben eines Genauigkeit-\#Epochen-Plots eine \emph{Confusionmatrix}
verwendet. Zusätzlich soll anhand der \emph{Loss}-Kurven vom Test- und Validierungsdatensatz
Overtraning festgemacht werden.

\subsection{Traningsweise}
Wie bereits im letzten Kaptiel erwähnt werden die
Bilder batchweise geladen und verarbeitet. Hierbei erfolgt auch eine Normierung
der Pixelwerte. Vor dem Beginn eines Tranings werden \emph{Batch-Größe} und
\emph{Learningrate} festgelegt. Letztere wird jedoch, während des Tranings dynamisch
durch die in \textsc{keras} implementierte Funtkion \textsc{ReduceLROnPlateau}
angepasst \cite{keras_ReduceLROnPlateau}. Die Anzahl der Traningsepochen
wird nicht festgelegt, da hier eine \emph{EarlyStopper} eingestetzt wird \cite{keras_EarlyStopping}.
Zusätzlich muss die Anzahl angegeben werden ob $5$ oder $120$ Hunderassen
klassifiziert werden sollen.

\subsection{NN für 5 Hunderassen}

\subsubsection{MiniDogNN}

\subsubsection{MiniDogNN -- Hyperparamter Optimerung}

\subsubsection{PreDogNN}


\subsection{NN für 120 Hunderassen}

\subsubsection{MiniDogNN}

\subsubsection{PreBigDogNN}
