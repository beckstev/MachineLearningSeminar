\documentclass[titlepage=firstiscover, bibliography=totoc, captions=tableheading, parskip=half]{scrbook}
\titlehead{
  \centering
  \includegraphics[scale=2.3]{pics/logo.png}
}
\author{Felix Geyer \\
  \texorpdfstring{\href{mailto:felix.geyer@tu-dortmund.de}{felix.geyer@tu-dortmund.de}}{}
  }
\title{Projektbericht zum Seminar: Maschinelles Lernen}
\publishers{TU Dortmund - Fakultät Physik}
\date{Abgabe: 15. Februar 2019}
\usepackage[aux]{rerunfilecheck}
\usepackage{polyglossia}
\setmainlanguage{german}
\usepackage{amsmath}
\usepackage{amssymb}
\usepackage{mathtools}
\usepackage{fontspec}
\usepackage[version=4]{mhchem}
\usepackage{scrhack}
\usepackage{float}
\floatplacement{table}{htbp}
\floatplacement{figure}{htbp}

\usepackage[locale=DE, separate-uncertainty=true, per-mode=reciprocal, decimalsymbol=comma]{siunitx}
%\usepackage{siunitx}
\DeclareSIUnit\px{px}

\usepackage[sorting=none]{biblatex}
\addbibresource{content/lit.bib}

\usepackage[section, below]{placeins}
\usepackage[labelfont=bf,
font=small,
width=0.9\textwidth,
format=plain,
indention=1em]{caption}
\usepackage{graphicx}
\usepackage{grffile}
\usepackage{subcaption}

\usepackage[math-style=ISO, bold-style=ISO, sans-style=italic, nabla=upright, partial=upright]{unicode-math}
\setmathfont{Latin Modern Math}

\usepackage[autostyle]{csquotes}

\usepackage[unicode]{hyperref}

\usepackage{mathtools}
\DeclarePairedDelimiter\abs{\lvert}{\rvert}

\usepackage{bookmark}

\usepackage{booktabs}

\usepackage{rotating}

\usepackage{tikz}
\usetikzlibrary{positioning}

\newcommand\JPsi{$J/\Psi$}
\newcommand\zerfall{$\beta$-Zerfall }
\newcommand\zerfalls{$\beta$-Zerfalls}
\newcommand\SppS{$\symup{Sp\bar{p}S}$}
\newcommand\barparen[1]{\overset{(-)}{#1}}

\begin{document}

\maketitle
  \section*{Inhaltsverzeichnis}
  \begin{frame}{Inhaltsverzeichnis}
    \tableofcontents
  \end{frame}

  \section{Definition und Motivation}
  \begin{frame}{Definition und Motivation}
    Fragestellung:
    \begin{center}
      \textbf{Wie schnell und präzise kann eine Hunderassen-Klassifizierung mit 120 Klassen
      durchgeführt werden?}
    \end{center}
    Motivation:
    \begin{itemize}
      \item Klassifizierung ohne Vorwissen möglich
      \item Persönliche Defizite ausgleichen
      \item Hunde sind cooler als Katzen
    \end{itemize}
  \end{frame}
  \section{Datensatz}
  \begin{frame}{Datensatz}
    \begin{itemize}
      \item \href{https://www.kaggle.com/jessicali9530/stanford-dogs-dataset}{Stanford Dogs Dataset}
      \item Lizenz: ???
      \item Informationen: Klassifikationen und Bounding Boxes
      \item Anzahl Einträge: 20.580 auf 120 Klassen verteilt
      \item Input-Features: Bilder
      \item Target-Variablen: Klassen und Bounding Boxes
      \item Bisherige Arbeiten: Kernels bei Kaggle
    \end{itemize}
  \end{frame}

  \section{Alternativ-Methode}
  \begin{frame}{Alternativ-Methode}
    Wir wollen zwei Architekturen vergleichen, ein CNN ohne Bounding Boxes und zwei
    CNN, bei welchen das erste die Bounding Boxes predicted und damit dann das zweite
    CNN ausführt, welches die Klassifizierung vornimmt.
  \end{frame}
\end{document}
