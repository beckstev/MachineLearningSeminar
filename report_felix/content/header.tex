\documentclass[titlepage=firstiscover, bibliography=totoc, captions=tableheading,
parskip=half, fontsize=12pt, left=2.5cm, right=2.5cm, top=3.5cm, bottom=3.5]{scrreprt}
\titlehead{
  \centering
  \includegraphics[scale=2.3]{pics/logo.png}
}
\author{Felix Geyer \\
  \texorpdfstring{\href{mailto:felix.geyer@tu-dortmund.de}{felix.geyer@tu-dortmund.de}}{}
  }
\title{Projektbericht zum Seminar: Maschinelles Lernen}
\publishers{TU Dortmund - Fakultät Physik}
\date{Abgabe: 31. Juli 2019}
\usepackage[aux]{rerunfilecheck}
\usepackage{polyglossia}
\setmainlanguage{german}
\usepackage{amsmath}
\usepackage{amssymb}
\usepackage{mathtools}
\usepackage{fontspec}
\usepackage[version=4]{mhchem}
\usepackage{scrhack}
\usepackage{float}
\floatplacement{table}{htbp}
\floatplacement{figure}{htbp}

\usepackage[locale=DE, separate-uncertainty=true, per-mode=reciprocal, decimalsymbol=comma]{siunitx}
%\usepackage{siunitx}
\DeclareSIUnit\px{px}

\usepackage[sorting=none]{biblatex}
\addbibresource{content/lit.bib}

\usepackage[section, below]{placeins}
\usepackage[labelfont=it,
font=small,
width=0.9\textwidth,
format=plain,
indention=1em]{caption}
\usepackage{graphicx}
\usepackage{grffile}
\usepackage{subcaption}

\usepackage[math-style=ISO, bold-style=ISO, sans-style=italic, nabla=upright, partial=upright]{unicode-math}
\setmathfont{Latin Modern Math}

\usepackage[autostyle]{csquotes}

\usepackage[unicode]{hyperref}

\usepackage{mathtools}
\DeclarePairedDelimiter\abs{\lvert}{\rvert}

\usepackage{bookmark}

\usepackage{booktabs}

\usepackage{rotating}

\usepackage{tikz}
\usetikzlibrary{positioning}

\AtBeginDocument{%
  \newcaptionname{ngerman}{\figureautorefname}{Abbildung}
  \newcaptionname{ngerman}{\tableautorefname}{Tabelle}
  \newcaptionname{ngerman}{\sectionautorefname}{Kapitel}
  \newcaptionname{ngerman}{\subsectionautorefname}{Kapitel}
  \newcaptionname{ngerman}{\chapterautorefname}{Kapitel}%
}

\newcommand\CNN{\texttt{CNN} }
\newcommand\RF{\texttt{RF} }
\newcommand\MiniDog{\texttt{MiniDogNN} }
\newcommand\PreDog{\texttt{PreDogNN} }
\newcommand\PreBig{\texttt{PreBigDogNN} }
\newcommand\barparen[1]{\overset{(-)}{#1}}

\usepackage{blindtext}
% For Times New Roman and 1,5 linespacing
\usepackage{mathptmx}
% \usepackage{newtxtext,newtxmath}
\linespread{1.25}
