\section{Zusammenfassung}
Zusammenfassend lässt sich sagen, dass die Farbinformationen der Bilder die
Klassifikation verbessern, so wie in \autoref{fig:hyper-param} sichtbar wurde.
Außerdem zeigte sich, dass die Nutzung von vortrainierten Netzen eine deutliche
Verbesserung der Genauigkeit liefert. Vor allem für den kleinen Datensatz wird
aus \autoref{fig:confusion-predog} ersichtlich, dass die Klassifikation sehr gut
funktioniert. Die Confusion-Matrizen von \MiniDog{} und \RF{} weisen noch große
Nebendiagonalelemente auf, die teilweise größer sind als das
Hauptdiagonalelement.

Weiterhin zeigte sich, dass die Klassifikation für den kleinen Datensatz besser
funktionierte als für den großen Datensatz. Zwar konnte auch für den großen
Datensatz eine deutliche Verbesserung der Genauigkeit durch die Verwendung eines
vortrainierten Netzes nachgewiesen werden, dieser ist allerdings nicht so groß
wie beim kleinen Datensatz. Allerdings muss bedacht werden, dass bei der Auswahl
des kleinen Datensatzes bewusst keine Unterarten gewählt wurden bzw. darauf
geachtet wurde, möglichst unterscheidbare Klassen zu wählen, um die
Klassifikation zu vereinfachen. Außerdem wurde deutlich, dass sich \MiniDog{}
nicht dazu eignet mit dem großen Datensatz zu arbeiten, wie
\autoref{fig:confusion-mini-120} zeigt.

Die Alternativ-Methode funktionierte besser als \MiniDog{}, ist aber wegen zu
wenig Parametern den vortrainierten Netzen unterlegen. Für den großen Datensatz
eignet sich die Alternativ-Methode besser als \MiniDog{}, was allerdings eher
darauf zurückzuführen ist, dass sich die Struktur von \MiniDog{} nicht für die
Klassifikation von so vielen Klassen eignet.

Zusammenfassend lässt sich vermuten, dass die Verwendung von \texttt{NN} im
Bereich Bildklassifikation zu präziseren Ergebnissen führt als anderen Machine Learning
Algorithmen, so wie in diesem Fall im Vergleich zu einem \RF{}. Außerdem bietet es sich an,
vortrainierte Netze zu nutzen, die auf Bildklassifikation trainiert
wurden, um auf diese Weise eine präzisere Klassifikation zu erhalten.
