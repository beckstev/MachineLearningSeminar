\chapter{Einleitung}
In diesem Projekt zum Seminar \enquote{Maschinelles Lernen} wurde ein
\texttt{Convolutional Neural Network (CNN)} trainiert, um 5 bzw. 120 verschiedene
Hunderassen voneinander zu unterscheiden. Der Datensatz besteht aus 20580
Bildern\cite{datensatz}. Die Motivation hinter diesem Projekt ist, eine Klassifikation von
Hunderassen ohne Vorwissen zu ermöglichen, da es einfacher ist, z.\,B. einen
Labrador von einem Pudel zu unterscheiden, aber die genauen Unterrassen des
Labradors oder des Terriers, welcher in unserem Datensatz mit 18 verschiedenen
Unterrassen vertreten ist, sind schwieriger zu unterscheiden, zumindest für die meisten
Menschen. Deswegen stellt sich die Frage, wie gut ein \CNN diese Aufgabe bewältigen kann.

Spezieller stellt sich auch die Frage, ob die Farbinformationen des Bildes zur
Klassifikation beitragen oder ob nur Formen entscheidend sind. Weiterhin wird
erörtert, inwiefern ein bereits vortrainiertes Netz zur Objektidentifizierung,
welches um einige Lagen erweitert wird, die Performance verbessert. Außerdem
soll erörtert werden, inwiefern sich die Performance der Klassifizierung des
großen Datensatzes mit 120 Klassen hinsichtlich des kleinen Datensatzes
unterscheidet und ob es vielleicht sogar möglich ist, dass \enquote{gleiche}
(mit ein paar kleinen Anpassungen) Netz für beide Klassifizierungen zu
verwenden.

Als Alternativ-Methode wird ein \texttt{Autoencoder} kombiniert mit einem
\texttt{Random Forest (RF)} verwendet. Dabei generiert der Autoencoder Features
aus den Bildern, die dann der Random Forest zur Klassifizierung nutzt. Als
\texttt{Performance Measure} wird unter anderem eine Confusion-Matrix verwendet.
Es wird verglichen, wie gut ein \RF im Vergleich zu einem \texttt{Neuronalen
Netz} Bildklassifikationen vornehmen kann, wenn er Features nutzt, die aus einem
Autoencoder generiert wurden.
