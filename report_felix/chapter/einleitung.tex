\section{Einleitung}
In diesem Projekt zum Seminar \enquote{Maschinelles Lernen} wurden mehrere
\texttt{Neuronale Netze (NN)} trainiert, um fünf bzw. 120 verschiedene
Hunderassen zu klassifizieren. Der Datensatz besteht aus 20580 Bildern
\cite{datensatz}. Die Motivation hinter diesem Projekt ist die Klassifikation
von Hunderassen mit ihren Unterrassen ohne Vorwissen zu ermöglichen.
Beispielsweise kann ein Mensch einfacher einen Labrador von einem Pudel
unterscheiden, als die Unterrassen des Terriers voneinander, von denen in diesem
Datensatz 18 verschiedene vertreten sind. Deswegen stellt sich die Frage, wie
gut ein \texttt{Neuronales Netz} diese Aufgabe bewältigen kann.

Spezieller stellt sich auch die Frage, ob die Farbinformationen des Bildes die
Klassifikation verbessern oder ob nur Formen entscheidend sind. Weiterhin wird
erörtert, inwiefern ein bereits vortrainiertes Netz zur Klassifikation beitragen
kann. Außerdem soll untersucht werden, inwiefern sich die Performance der
Klassifizierung des großen Datensatzes mit 120 Klassen hinsichtlich des kleinen
Datensatzes unterscheidet und ob es möglich ist, ein geringfügig modifiziertes
Netz für beide Klassifizierungen zu verwenden.

Als Alternativ-Methode wird ein \texttt{Autoencoder} kombiniert mit einem
\texttt{Random Forest (RF)} verwendet. Dabei generiert der Autoencoder Features
aus den Bildern, die dann der Random Forest zur Klassifizierung nutzt. Als
\texttt{Performance Measure} wird eine Confusion-Matrix verwendet.
Es wird verglichen, wie gut ein \RF{} im Vergleich zu einem \texttt{NN}
Bildklassifikationen vornehmen kann.
